\section{Conclusiones}
%RESUMIR a 2 Diapositivas
\begin{frame}[t]
\frametitle{Conclusiones}
\small
\begin{itemize}
	\item El sistema propuesto es una \textbf{solución IoT} funcional, ya que cuenta con la capacidad de \textbf{monitorear, controlar y programar} una \textbf{habitación} dentro de un entorno \textbf{Smart House}. No obstante, el sistema propuesto no se limita a este escenario, ya que está diseñado con el fin de abarcar un amplio número de \textbf{tareas}, además de que acepta múltiples tipos de dispositivos de medida.
	
	\item La \textbf{aplicación web} desarrollada cumple con los diferentes alcances, permitiendo la \textbf{interacción} de los \textbf{usuarios} con los diferentes \textbf{dispositivos} presentes en su entorno a través de su \textbf{interfaz}, además de que esta puede seguir siendo ampliada con el fin de adicionar funcionalidades al sistema.
	
	\item El \textbf{hardware} está diseñado en dos \textbf{etapas} con un circuito para cada una, permitiendo una clasificación en cuanto a su funcionamiento, además de representar mayor comodidad en cuanto a diseño de \textbf{prototipo}, evitando que los altos voltajes de la etapa AC causen algún tipo de interferencia de manera directa con la etapa DC.
	
\end{itemize}
\end{frame}

\begin{frame}[t]
\frametitle{Conclusiones}
\begin{itemize}
	\item La tarjeta de prototipado \textbf{ESP32} como unidad central de procesamiento dentro del desarrollo del hardware, representa una opción viable en la puesta en funcionamiento de un \textbf{sistema IoT}, no solo por su \textbf{bajo costo} sino también por sus características y funcionalidades, permitiendo que el diseño y la implementación del sistema sea \textbf{escalable}.
	
	\item El uso del \textbf{framework ESP-IDF}, que se compone de un sistema operativo en tiempo real \textbf{freeRTOS}, es muy útil para realizar las diferentes funciones del sistema que comunica hardware con software, además de que permite ejecutar las diversas tareas y gestionar bien los \textbf{recursos} presentes en el chip \textbf{ESP32}, por lo tanto es posible agregar una amplia gama de características adicionales a la funcionalidad del sistema.
\end{itemize}
\end{frame}