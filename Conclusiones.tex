\section{Conclusiones}
%RESUMIR a 2 Diapositivas
\begin{frame}
	El sistema propuesto es una solución IoT funcional, ya que cuenta con la capacidad de monitorear, controlar y programar una habitación dentro de un entorno Smart House. No obstante, el sistema propuesto no se limita a este escenario, ya que está diseñado con el fin de abarcar un amplio número de tareas, además de que acepta múltiples tipos de dispositivos de medida.\newline
	
	La aplicación web desarrollada cumple con los diferentes alcances, permitiendo la interacción de los usuarios con los diferentes dispositivos presentes en su entorno a través de su interfaz, además de que esta puede seguir siendo ampliada con el fin de adicionar funcionalidades al sistema.\newline
	
	El hardware está diseñado en dos etapas con un circuito para cada una, permitiendo una clasificación en cuanto a su funcionamiento, además de representar mayor comodidad en cuanto a diseño de prototipo, evitando que los altos voltajes de la etapa AC causen algún tipo de interferencia de manera directa con la etapa DC. \\
\end{frame}

\begin{frame}
	La tarjeta de prototipado ESP32 como unidad central de procesamiento dentro del desarrollo del hardware, representa una opción viable en la puesta en funcionamiento de un sistema IoT, no solo por su bajo costo sino también por sus características y funcionalidades, permitiendo que el diseño y la implementación del sistema sea escalable.\newline
	
	El uso del framework ESP-IDF, que se compone de un sistema operativo en tiempo real freeRTOS, es muy útil para realizar las diferentes funciones del sistema que comunica hardware con software, además de que permite ejecutar las diversas tareas y gestionar bien los recursos presentes en el chip ESP32, por lo tanto es posible agregar una amplia gama de características adicionales a la funcionalidad del sistema.\\
	
\end{frame}