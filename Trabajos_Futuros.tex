\section{Trabajos Futuros}

\begin{frame}[t]
\frametitle{Trabajos Futuros}

Como trabajo futuro en cuanto al hardware se propone integrar todo el sistema en una sola board, migrando la mayoría de componentes de montaje through hole a superficial (SMD) y también realizar un diseño con más capas, por ejemplo de cuatro capas, para reducir el tamaño de esta e incorporar todas las funcionalidades en una sola, además de añadir mas características a este, como el manejo de otros dispositivos por medio de infrarrojo, poder actualizar su firmware remotamente y funciones de ahorro de energía.\newline

En el software es conveniente agregar funcionalidades de acuerdo con los otros roles presentes en este, por ejemplo para el rol usuario administrador de casa crear reglas de control parental y de seguridad con el fin de ampliar la administración sobre sus propios dispositivos.\\

\end{frame}

\begin{frame}[t]
\frametitle{Trabajos Futuros}

En la parte de realización de la prueba del sistema se pueden realizar otros tipos como alfa, de aceptación, entre otras más que existen para probar los diferentes software diseñados, además ampliar la muestra en la que se aplican estas pruebas.\newline

Para mejorar la solución IoT es posible crear servidores espejo localmente en la tarjeta incluyéndolo en el firmware y realizando copias de la información constantemente. Sus funcionalidades son en caso de mantenimiento de la aplicación principal o de cualquier problema relativo a la conexión a Internet, esta daría soporte local hasta que entre de nuevo en funcionamiento la que se encuentra en la nube.\\
\end{frame}