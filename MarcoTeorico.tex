\section{Marco Teórico}

\begin{frame}[t]
\subsection{Internet de las Cosas}
\frametitle{Marco Teórico}

\small
\textbf{La Internet de las Cosas} es un sistema de dispositivos de computación interrelacionados, máquinas mecánicas y digitales, objetos, animales o personas que tienen identificadores únicos y la capacidad de transferir datos a través de una red, sin requerir de interacciones humano a humano o humano a computadora. \newline

\textbf{Smart House} es un entorno que tiene sistemas sofisticados a través de los cuales se pueden controlar algunos de los objetos de la casa, como luces, puertas, ventanas, además puede racionalizar el consumo de energía, entre otras funciones mediante el uso de sensores. Básicamente, uno de los beneficios más importantes del uso de la tecnología en las casas, es la prestación de servicios a las personas.\cite{Howedi2016}\newline

Una \textbf{habitación (room)} en este entorno de trabajo hace referencia a cualquier lugar de la casa, por ejemplo, el baño (bathroom), la sala (living room), etc.
\end{frame}